---
layout: page
title: "Axiom of Univalence"
category: lemmas functions homotopy
toc: true
agda: true
gallery: true
latex: true
references: false
linkify: true
showcitation: true
---

<div class="hide" >
\begin{code}%
\>[0]\AgdaSymbol{\{-\#}\AgdaSpace{}%
\AgdaKeyword{OPTIONS}\AgdaSpace{}%
\AgdaPragma{--without-K}\AgdaSpace{}%
\AgdaSymbol{\#-\}}\<%
\\
\>[0]\AgdaKeyword{open}\AgdaSpace{}%
\AgdaKeyword{import}\AgdaSpace{}%
\AgdaModule{TransportLemmas}\<%
\\
\>[0]\AgdaKeyword{open}\AgdaSpace{}%
\AgdaKeyword{import}\AgdaSpace{}%
\AgdaModule{EquivalenceType}\<%
\\
%
\\[\AgdaEmptyExtraSkip]%
\>[0]\AgdaKeyword{open}\AgdaSpace{}%
\AgdaKeyword{import}\AgdaSpace{}%
\AgdaModule{HomotopyType}\<%
\\
\>[0]\AgdaKeyword{open}\AgdaSpace{}%
\AgdaKeyword{import}\AgdaSpace{}%
\AgdaModule{FunExtAxiom}\<%
\\
\>[0]\AgdaKeyword{open}\AgdaSpace{}%
\AgdaKeyword{import}\AgdaSpace{}%
\AgdaModule{QuasiinverseType}\<%
\end{code}
</div>

## Voevodsky's Univalence Axiom

Voevodsky's Univalence axiom is postulated. It induces an equality between any
two equivalent types. Some $β$ and $η$ rules are provided.

\begin{code}%
\>[0]\AgdaKeyword{module}\AgdaSpace{}%
\AgdaModule{UnivalenceAxiom}\AgdaSpace{}%
\AgdaSymbol{\{}\AgdaBound{ℓ}\AgdaSymbol{\}}\AgdaSpace{}%
\AgdaSymbol{\{}\AgdaBound{A}\AgdaSpace{}%
\AgdaBound{B}\AgdaSpace{}%
\AgdaSymbol{:}\AgdaSpace{}%
\AgdaFunction{Type}\AgdaSpace{}%
\AgdaBound{ℓ}\AgdaSymbol{\}}\AgdaSpace{}%
\AgdaKeyword{where}\<%
\end{code}

{: .foldable until="4"}
\begin{code}%
\>[0][@{}l@{\AgdaIndent{1}}]%
\>[2]\AgdaFunction{idtoeqv}\<%
\\
\>[2][@{}l@{\AgdaIndent{0}}]%
\>[4]\AgdaSymbol{:}\AgdaSpace{}%
\AgdaBound{A}\AgdaSpace{}%
\AgdaOperator{\AgdaDatatype{==}}\AgdaSpace{}%
\AgdaBound{B}\<%
\\
%
\>[4]\AgdaComment{--------}\<%
\\
%
\>[4]\AgdaSymbol{→}\AgdaSpace{}%
\AgdaBound{A}\AgdaSpace{}%
\AgdaOperator{\AgdaFunction{≃}}\AgdaSpace{}%
\AgdaBound{B}\<%
\\
%
\\[\AgdaEmptyExtraSkip]%
%
\>[2]\AgdaFunction{idtoeqv}\AgdaSpace{}%
\AgdaBound{p}\AgdaSpace{}%
\AgdaSymbol{=}\<%
\\
\>[2][@{}l@{\AgdaIndent{0}}]%
\>[4]\AgdaFunction{qinv-≃}\<%
\\
\>[4][@{}l@{\AgdaIndent{0}}]%
\>[6]\AgdaSymbol{(}\AgdaFunction{coe}\AgdaSpace{}%
\AgdaBound{p}\AgdaSymbol{)}\<%
\\
%
\>[6]\AgdaSymbol{((}\AgdaFunction{!coe}\AgdaSpace{}%
\AgdaBound{p}\AgdaSymbol{)}\AgdaSpace{}%
\AgdaOperator{\AgdaInductiveConstructor{,}}\<%
\\
\>[6][@{}l@{\AgdaIndent{0}}]%
\>[8]\AgdaSymbol{(}\AgdaFunction{coe-inv-l}\AgdaSpace{}%
\AgdaBound{p}\AgdaSpace{}%
\AgdaOperator{\AgdaInductiveConstructor{,}}\AgdaSpace{}%
\AgdaFunction{coe-inv-r}\AgdaSpace{}%
\AgdaBound{p}\AgdaSymbol{))}\<%
\end{code}

Synonyms:

\begin{code}%
%
\>[2]\AgdaFunction{==-to-≃}\AgdaSpace{}%
\AgdaSymbol{=}\AgdaSpace{}%
\AgdaFunction{idtoeqv}\<%
\\
%
\>[2]\AgdaFunction{≡-to-≃}%
\>[10]\AgdaSymbol{=}\AgdaSpace{}%
\AgdaFunction{idtoeqv}\<%
\\
%
\>[2]\AgdaFunction{ite}%
\>[10]\AgdaSymbol{=}\AgdaSpace{}%
\AgdaFunction{idtoeqv}\<%
\end{code}

The **Univalence axiom** induces an equivalence between equalities
and equivalences.

Univalence Axiom.

\begin{code}%
%
\>[2]\AgdaKeyword{postulate}\<%
\\
\>[2][@{}l@{\AgdaIndent{0}}]%
\>[4]\AgdaPostulate{axiomUnivalence}\<%
\\
\>[4][@{}l@{\AgdaIndent{0}}]%
\>[6]\AgdaSymbol{:}\AgdaSpace{}%
\AgdaFunction{isEquivalence}\AgdaSpace{}%
\AgdaFunction{≡-to-≃}\<%
\end{code}

In Slide 20 from an [Escardo's talk](https://www.newton.ac.uk/files/seminar/20170711100011001-1442677.pdf), base on what we saw, we give
the following no standard definition of Univalence axiom (without transport).

\begin{code}%
%
\>[2]\AgdaFunction{UA}\<%
\\
\>[2][@{}l@{\AgdaIndent{0}}]%
\>[4]\AgdaSymbol{:}\AgdaSpace{}%
\AgdaSymbol{∀}\AgdaSpace{}%
\AgdaSymbol{\{}\AgdaBound{ℓ}\AgdaSymbol{\}}\<%
\\
%
\>[4]\AgdaSymbol{→}\AgdaSpace{}%
\AgdaSymbol{(}\AgdaFunction{Type}\AgdaSpace{}%
\AgdaSymbol{(}\AgdaPrimitive{lsuc}\AgdaSpace{}%
\AgdaBound{ℓ}\AgdaSymbol{))}\<%
\\
%
\\[\AgdaEmptyExtraSkip]%
%
\>[2]\AgdaFunction{UA}\AgdaSpace{}%
\AgdaSymbol{\{}\AgdaArgument{ℓ}\AgdaSpace{}%
\AgdaSymbol{=}\AgdaSpace{}%
\AgdaBound{ℓ}\AgdaSymbol{\}}%
\>[14]\AgdaSymbol{=}\<%
\\
\>[2][@{}l@{\AgdaIndent{0}}]%
\>[4]\AgdaSymbol{(}\AgdaBound{X}\AgdaSpace{}%
\AgdaSymbol{:}\AgdaSpace{}%
\AgdaFunction{Type}\AgdaSpace{}%
\AgdaBound{ℓ}\AgdaSymbol{)}\AgdaSpace{}%
\AgdaSymbol{→}\AgdaSpace{}%
\AgdaFunction{isProp}\AgdaSpace{}%
\AgdaSymbol{(}\AgdaSpace{}%
\AgdaRecord{∑}\AgdaSpace{}%
\AgdaSymbol{(}\AgdaFunction{Type}\AgdaSpace{}%
\AgdaBound{ℓ}\AgdaSymbol{)}\AgdaSpace{}%
\AgdaSymbol{(λ}\AgdaSpace{}%
\AgdaBound{Y}\AgdaSpace{}%
\AgdaSymbol{→}\AgdaSpace{}%
\AgdaSymbol{(}\AgdaBound{X}\AgdaSpace{}%
\AgdaOperator{\AgdaFunction{≃}}\AgdaSpace{}%
\AgdaBound{Y}\AgdaSymbol{)}\AgdaSpace{}%
\AgdaSymbol{))}\<%
\\
%
\>[4]\AgdaKeyword{where}\AgdaSpace{}%
\AgdaKeyword{open}\AgdaSpace{}%
\AgdaKeyword{import}%
\>[23]\AgdaModule{HLevelTypes}\<%
\end{code}

About this Univalence axiom version:

  - ∑ (Type ℓ) (λ Y → X ≃ Y) is inhabited, but we don't know if it's contractible
  unless, we demand (assume) to be propositional. Then, in such a case,
  we use the theorem (isProp P ≃ (P → isContr P)). To be more precise, we know it's contractible, in fact, the center of contractibility, is indeed (X, id-≃ X : X ≃ X).

  - Univalence is a generalized extensionality axiom for intensional MLTT theory.
  - The type UA is a proposition.
  - UA is consistent with MLTT.
  - Theorem of MLTT+UA: $P(X)$ and $X≃Y$ imply $P(Y)$ for any $P : \mathsf{Type} → \mathsf{Type}$.
  - Theorem of spartan MLTT with two universes. The univalence axiom formulated
with crude isomorphism rather than equivalence is false!.

{: .foldable until="2" }
\begin{code}%
%
\>[2]\AgdaFunction{eqvUnivalence}\<%
\\
\>[2][@{}l@{\AgdaIndent{0}}]%
\>[4]\AgdaSymbol{:}\AgdaSpace{}%
\AgdaSymbol{(}\AgdaBound{A}\AgdaSpace{}%
\AgdaOperator{\AgdaDatatype{==}}\AgdaSpace{}%
\AgdaBound{B}\AgdaSymbol{)}\AgdaSpace{}%
\AgdaOperator{\AgdaFunction{≃}}\AgdaSpace{}%
\AgdaSymbol{(}\AgdaBound{A}\AgdaSpace{}%
\AgdaOperator{\AgdaFunction{≃}}\AgdaSpace{}%
\AgdaBound{B}\AgdaSymbol{)}\<%
\\
%
\\[\AgdaEmptyExtraSkip]%
%
\>[2]\AgdaFunction{eqvUnivalence}\AgdaSpace{}%
\AgdaSymbol{=}\AgdaSpace{}%
\AgdaFunction{idtoeqv}\AgdaSpace{}%
\AgdaOperator{\AgdaInductiveConstructor{,}}\AgdaSpace{}%
\AgdaPostulate{axiomUnivalence}\<%
\end{code}

Synonyms:
\begin{code}%
%
\>[2]\AgdaFunction{==-equiv-≃}\AgdaSpace{}%
\AgdaSymbol{=}\AgdaSpace{}%
\AgdaFunction{eqvUnivalence}\<%
\\
%
\>[2]\AgdaFunction{==-≃-≃}%
\>[13]\AgdaSymbol{=}\AgdaSpace{}%
\AgdaFunction{eqvUnivalence}\<%
\\
%
\>[2]\AgdaFunction{≡-≃-≃}%
\>[13]\AgdaSymbol{=}\AgdaSpace{}%
\AgdaFunction{eqvUnivalence}\<%
\end{code}

Introduction rule for equalities:

{: .foldable until="4" }
\begin{code}%
%
\>[2]\AgdaFunction{ua}\<%
\\
\>[2][@{}l@{\AgdaIndent{0}}]%
\>[4]\AgdaSymbol{:}\AgdaSpace{}%
\AgdaBound{A}\AgdaSpace{}%
\AgdaOperator{\AgdaFunction{≃}}\AgdaSpace{}%
\AgdaBound{B}\<%
\\
%
\>[4]\AgdaComment{-------}\<%
\\
%
\>[4]\AgdaSymbol{→}\AgdaSpace{}%
\AgdaBound{A}\AgdaSpace{}%
\AgdaOperator{\AgdaDatatype{==}}\AgdaSpace{}%
\AgdaBound{B}\<%
\\
%
\\[\AgdaEmptyExtraSkip]%
%
\>[2]\AgdaFunction{ua}\AgdaSpace{}%
\AgdaSymbol{=}\AgdaSpace{}%
\AgdaFunction{remap}\AgdaSpace{}%
\AgdaFunction{eqvUnivalence}\<%
\end{code}

Synonyms:

\begin{code}%
%
\>[2]\AgdaFunction{≃-to-==}%
\>[12]\AgdaSymbol{=}\AgdaSpace{}%
\AgdaFunction{ua}\<%
\\
%
\>[2]\AgdaFunction{eqv-to-eq}\AgdaSpace{}%
\AgdaSymbol{=}\AgdaSpace{}%
\AgdaFunction{ua}\<%
\end{code}

Computation rules

{: .foldable until="4"}
\begin{code}%
%
\>[2]\AgdaFunction{ua-β}\<%
\\
\>[2][@{}l@{\AgdaIndent{0}}]%
\>[4]\AgdaSymbol{:}\AgdaSpace{}%
\AgdaSymbol{(}\AgdaBound{α}\AgdaSpace{}%
\AgdaSymbol{:}\AgdaSpace{}%
\AgdaBound{A}\AgdaSpace{}%
\AgdaOperator{\AgdaFunction{≃}}\AgdaSpace{}%
\AgdaBound{B}\AgdaSymbol{)}\<%
\\
%
\>[4]\AgdaComment{----------------------}\<%
\\
%
\>[4]\AgdaSymbol{→}\AgdaSpace{}%
\AgdaFunction{idtoeqv}\AgdaSpace{}%
\AgdaSymbol{(}\AgdaFunction{ua}\AgdaSpace{}%
\AgdaBound{α}\AgdaSymbol{)}\AgdaSpace{}%
\AgdaOperator{\AgdaDatatype{==}}\AgdaSpace{}%
\AgdaBound{α}\<%
\\
%
\\[\AgdaEmptyExtraSkip]%
%
\>[2]\AgdaFunction{ua-β}\AgdaSpace{}%
\AgdaBound{eqv}\AgdaSpace{}%
\AgdaSymbol{=}\AgdaSpace{}%
\AgdaFunction{lrmap-inverse}\AgdaSpace{}%
\AgdaFunction{eqvUnivalence}\<%
\end{code}

{: .foldable until="4"}
\begin{code}%
%
\>[2]\AgdaFunction{ua-η}\<%
\\
\>[2][@{}l@{\AgdaIndent{0}}]%
\>[4]\AgdaSymbol{:}\AgdaSpace{}%
\AgdaSymbol{(}\AgdaBound{p}\AgdaSpace{}%
\AgdaSymbol{:}\AgdaSpace{}%
\AgdaBound{A}\AgdaSpace{}%
\AgdaOperator{\AgdaDatatype{==}}\AgdaSpace{}%
\AgdaBound{B}\AgdaSymbol{)}\<%
\\
%
\>[4]\AgdaComment{---------------------}\<%
\\
%
\>[4]\AgdaSymbol{→}\AgdaSpace{}%
\AgdaFunction{ua}\AgdaSpace{}%
\AgdaSymbol{(}\AgdaFunction{idtoeqv}\AgdaSpace{}%
\AgdaBound{p}\AgdaSymbol{)}\AgdaSpace{}%
\AgdaOperator{\AgdaDatatype{==}}\AgdaSpace{}%
\AgdaBound{p}\<%
\\
%
\\[\AgdaEmptyExtraSkip]%
%
\>[2]\AgdaFunction{ua-η}\AgdaSpace{}%
\AgdaBound{p}\AgdaSpace{}%
\AgdaSymbol{=}\AgdaSpace{}%
\AgdaFunction{rlmap-inverse}\AgdaSpace{}%
\AgdaFunction{eqvUnivalence}\<%
\end{code}
